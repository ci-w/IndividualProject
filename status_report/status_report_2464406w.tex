    
\documentclass[11pt]{article}
\usepackage{times}
    \usepackage{fullpage}
    
    \title{ Dynamic Making Guides }
    \author{ Claire Williamson - 2464406W }

\begin{document}
\maketitle




\section{Status report}

\subsection{Proposal}\label{proposal}

\subsubsection{Motivation}\label{motivation}

Makerspaces and the community surrounding them have grown in popularity in recent years. These spaces aim to be a place for people to create physical objects by sharing tools, experience, and support. A large component of this community is focused on education and inclusivity, especially for marginalised groups. However there is a discrepancy between this idea and the reality. Many disabled people would greatly benefit from participating yet barriers prevent them, including tutorials simply being inaccessible for their needs. There's a knowledge gap between the different groups: makerspaces generally have little knowledge of disability, and disability support groups have little knowledge of making projects. Making projects can be used to educate on many topics including computer science such as programming and robotics ideas. They also use computing science to carry out this education in the forms of HCI/HRI and the different technologies used. 


\subsubsection{Aims}\label{aims}

This project aims to create a website that hosts making project tutorials that dynamically adapt to the skills and abilities of the user. It will allow a user to specify an end goal project and then create a path that teaches them the needed skills in a way accessible and engaging to them. Focused mainly at users with cognitive impairments such as intellectual disabilities and brain injuries. Projects will be able to be embedded into other sites. It aims it to be a tool to bridge the gap between disabled people and the benefits makerspaces provide. 

\subsection{Progress}\label{progress}

\begin{itemize}
    \item Reviewed relevant literature (30+ papers) surrounding makerspaces, tangible learning for people with disabilities, accessible makerspaces, DIY assistive technology. 
    \item Reviewed current DIY tutorial sites (5) to see strengths and failings as well as types of projects.
    \item Identified 6 local organisations to contact, from a range of makerspaces, intellectual disability organisations, and brain injury organisations. 
    \item Created website with static tutorials and the structures needed for the additional features.
    \item Hosted website online.
    \item Investigated ethics process. 
    \item Created 2 plans for ethics: getting College approval for evaluation including people with cognitive impairments, and completing the standard ethics checklist for evaluation not including them. 
\end{itemize}

\subsection{Problems and risks}\label{problems-and-risks}

\subsubsection{Problems}\label{problems}

\begin{itemize}
    \item Issue with ethics advice. Had trouble figuring out what ethical process I'd have to go through to include people with intellectual disabilities/brain injuries. Got conflicting advice, took a few emails to different people.
    \item Issue with ethics approval. Found out being able to get input on the project from the intended users would require a lengthy ethical approval process that has a chance of being declined, with a timescale of several weeks. 
    \item Creating the basic website took longer than planned.
    \item When hosting the website online, ran into several problems that took a while to fix regarding adapting the project to the new environment.  
\end{itemize}

\subsubsection{Risks}\label{risks}

\begin{itemize}
    \item Not getting ethical approval for people with intellectual disabilities/brain injuries. \textbf{Mitigation}: instead talk to people have knowledge of such groups, which would comply with the standard ethics checklist. 
    \item Not finding any participants from the organisations. \textbf{Mitigation}: evaluate based on criteria from research papers of similar topics. 
\end{itemize}

\subsection{Plan}\label{plan}

\begin{itemize}
    \item Before semester 2: \textbf{Deliverables}: contact email scripts, finished lit review, website that allows profile creation and modification.
    \item Week 1: Contact organisations, add "syllabus" functionality to website. \textbf{Deliverables}: user can select a project, and gets a route of projects, using their current skills/abilities,culminating in the chosen project. 
    \item Week 2: Collect initial feedback about project from organisations. Implement dynamic project view. Make decision on ethics approval process. \textbf{Deliverables}: projects are displayed based on user's given ability. 
    \item Week 3: Integrate syllabus and dynamic projects. \textbf{Deliverables}: projects given in a syllabus depend on the users ability.  
    \item Weeks 4-5: Collate project tutorials, carry out some to gain in-depth knowledge/"case studies". \textbf{Deliverables}: detailed notes on some projects. 
    \item Weeks 5-6: Implement feedback system for projects. \textbf{Deliverables}: users can submit feedback after completing each project + users can see the opinions of other people with similar ability.
    \item Week 7: Write test suite and final evaluation plan for project. \textbf{Deliverables}: test suite that covers each section and their interactions, detailed evaluation plan including information/debriefing sheet and how the data will be analysed. 
    \item Week 8: Finish implementation, taking into account any feedback from organisations. \textbf{Deliverables}: polished website that passes test suite.
    \item Week 9: Evaluation experiments run. \textbf{Deliverables}: data from the participants on site usability, design, and efficiency.
    \item Week 5-10: Write up. \textbf{Deliverables}: draft submitted to supervisor 2 weeks before final deadline. 
\end{itemize}

\subsection{Ethics and data}\label{ethics}

I have sought ethical guidance from the School's ethics convener and I will: 
    \begin{itemize}
        \item Apply for College Ethics Board approval, for evaluation with participants who have intellectual disabilities/brain injuries.    
    \end{itemize}

Additionally, for evaluation not including these groups I have verified that the ethics checklist will apply. I have signed and completed the checklist, which has been signed by my supervisor. 

I expect to collect data on opinions surrounding the design, usability and feasibility of the site by its intended users. 


\end{document}
